%----------------------------------------------------------------------------
\chapter{\bevezetes}
%----------------------------------------------------------------------------

\section*{Témaválasztás indoklása}\addcontentsline{toc}{section}{Témaválasztás indoklása}

Beágyazott rendszerekre való fejlesztés során gyakran csak utólagosan merül fel a tesztek írásának gondolata, amelynek hatására a fejlesztés nagy része debuggolással fog elmenni, régi funkciókat gyakran rontunk el új funkciók beépítésével, Unit tesztelés hiánya miatt nincsenek jól elhatárolt szoftverkomponensek, emiatt keverednek a felelősségek a létező szoftverkomponensek között. Nem véletlen, hogy a biztonság kritikus rendszereket szabályzó szabványok, mint az ISO-26262 vagy DO-178C, megkövetelik a szoftverkomponensek Unit tesztelését. \cite{Szakdolgozat} 

Előző félévekben készült munkákban több, a beágyozott szoftvert fejlesztők munkáját megkönnyítő megoldás készült: egy unit test keretrendszer, Python nyelven írodott Mock generátor és egy VSCode extension, ami megegyszerűsíti egy projekt felépítését, azzal, hogy egy Meson build rendszer vázat generál, amiben már automatikusan működik a unit tesztek fordítása és futtatása a hoszt gépen. A célhardveren való futtatáshoz még egy plusz munkára van szükség a felhasználótól, de ez is lehetséges, csak a mikrokontroller és a futtató PC közötti kommunikációt kell megoldani, erre egy példát is mutattam a JLink debugger RTT funkciójával \cite{SeggerRTT}, ami egy memórián keresztüli kommunikációt valósít meg.

A kész megoldással már lehetséges a saját fejlesztői környezetben gyorsan és egyszerűen teszteket írni és futtatni, de a legtöbb projekt, ahol indokolt az ilyen típusú szoftverfejlesztés, ott nem egyedül történik a fejlesztés. Ez gyakorlatban azt jelenti, hogy egy a projekt kódbázisát, hozzá tartozó teszteket egy verziókezelőn tároljuk és használjuk, például Git-en vagy SVN-en. Felmerülhet az igény, hogy minden változtatásra a szoftverben, vagyis minden commit esetén futtassuk le a megírt teszteket, hogy megbizonyosudjunk arról, hogy a változtatásnak nincsen nem várt hatása. Ezt hívják Continous Integration-nek (CI). A beágyozott szoftverfejlesztéskor tehát a következő igényeink lehetnek, hogy a CI milyen feladatokat végezzen el:
\begin{itemize}
	\item Unit tesztek futtatása hoszt környezetben
	\item Unit tesztek futtatása célhardveren
	\item Kódanalizátorok futtatása
	\item Beágyazott eszközre kerülő szoftver fordítása
	\item Funkcionális tesztek futtatása az előzően fordított szoftverre 
\end{itemize}

A fentiek közül a hardvert nem igénylő feladatok könnyen megoldhatóak, akár párhuzamosan is külön Docker konténerekben. A célhardveren való tesztfuttatás viszont már nagyobb fejfájást okozhat a fejlesztőknek. Itt több feladatot is meg kell oldani. Először is a tesztprogram letöltését a hardverre. Ez az előző megoldás alapján egyszerűen megtehető egy debugger használatával. A következő probléma az eredmények kiolvasása, ami ugyanúgy megoldható debugger-el. 

Első látszatra ez elegendőnek tűnhet, és kisebb projektek esetén meg is felelhet, viszont több probléma is felmerülhet. Az eredeti ötlet az volt, hogy lehetőleg minden változásra fussanak a tesztek. Nagyobb projekteknél, vagy akár céges szinten nagyon sok ilyen tesztfuttatás történne, és ahhoz hogy ezek a tesztfuttatások párhuzamosan történjenek, sok tesztfuttató hardver szükséges, hogy ne kelljen ezek eredményére sokat várni. Ez azt jelenti, hogy a tesztfuttató hardverenként egy debuggert is meg kell vásárolni. Néhány architektúra esetében ez nagyon költséges lenne, akár több ezer euró is lehet egy debugger, licensz árakat nem is említve (például Lauterbach az Infineon TriCore architektúrához).

A másik probléma, hogy ha nem teszünk külön lépéseket, akkor futtatás közbeni hibákról, mint egy kivétel, nagyon kevés információ lesz elérhető. Emiatt szükség van egy olyan megoldásra, ami mindent megtesz, hogy egy elszigetelt környezetben fusson a teszt, és hiba esetén képes legyen jelezni a hiba okát.

\section*{Célkitűzés}\addcontentsline{toc}{section}{Célkitűzés}

A dolgozat célja tehát egy olyan környezet létrehozása, ami képes tetszőleges kommunikációs interfészen fogadni tesztprogramot (bootloader), meg tudja különböztetni a teszt és futtatókörnyezetben történő hibákat, és erről hibajelzést is tud adni a kliens felé. Ez azt is jelenti, hogy egy kliens alkalmazásnak is kell léteznie, amin keresztül a felhasználó tud interaktálni a tesztfuttató hardverrel.